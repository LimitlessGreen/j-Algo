\subsection{Anzeige von EBNF-Defintionen}

Die EBNF-Anzeige kann nur erreicht werden, wenn die Definition korrekt und vollst�ndig ist. Eine �berpr�fung dessen findet durch die vorausgegangene Nutzung des \textsc{Eingabe beenden}-Buttons bzw. Beim Laden einer Datei statt. \\

Hintergrund der EBNF-Anzeige ist die �bersichtliche Anzeige einer EBNF-Definition, ohne die Eingabemaske. Au�erdem besteht hier die M�glichkeit, zum trans()-Algorithmus zu wechseln, um die EBNF-Definition in ein Syntaxdiagramm umzuwandeln. 

\subsubsection{Definition bin�r klammern}

Um in den trans()-Algorithmus wechseln zu k�nnen, muss die Definition bin�r geklammert sein. Nach strenger Auslegung der Definition existieren nur bin�re Alternativen in der EBNF: 
\begin{itemize}
	\item nicht bin�r geklammert: (a|b|c)
	\item korrekt geklammert: (a|(b|c))
\end{itemize}

\centerpic{ebnf/ebnfdisplay_makebinary.png}{.6}{Definition bin�r klammern}

\bigskip

Der Lesbarkeit halber erm�glicht das Programm eine Eingabe von nicht bin�r geklammerten Alternativen. Ob die Definition noch nicht korrekt geklammerte Terme enth�lt, wird durch eine entsprechende Meldung angezeigt: BILD \\

Um eine Definition bin�r zu klammern, muss der \textsc{Definition bin�r klammern}-Button genutzt werden. Dadurch werden die noch fehlenden Klammern hinzugef�gt und rot hervorgehoben. Damit ist der Wechsel zum trans()-Algorithmus erm�glicht. 

\subsubsection{Wechsel zum trans()-Algorithmus}

Wenn die Definition bin�r geklammert ist, kann man in den trans()-Algorithmus durch Anklicken des \textsc{trans()-Algorithmus anwenden}-Buttons wechseln.

\subsubsection{Beibehalten der bin�ren Klammerung}

Falls in der EBNF-Anzeige nachtr�glich eine bin�re Klammerung vorgenommen wurde, so wird beim Wechsel zur EBNF-Eingabe (mittels \textsc{Definition �berarbeiten}-Button) und beim Speichern aus der EBNF-Anzeige heraus nachgefragt, ob diese in der Definition beibehalten werden soll.

\bigskip
\centerpic{ebnf/ebnfdisplay_keepbinary.png}{.8}{Beibehalten der Klammerung}
\bigskip