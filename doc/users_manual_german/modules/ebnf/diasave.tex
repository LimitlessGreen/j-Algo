\subsection{Speichern und Laden von Syntaxdiagrammen}

\subsubsection{Speichern}

Nat�rlich haben Sie auch die M�glichkeit, Syntaxdiagramme zu speichern. \\

Dies ist m�glich in der Syntaxdiagramm-Anzeige sowie im Syntaxdiagramm-Editor. Das Speichern ist nur mit der Endung \emph{"`*.jalgo"'} m�glich. Das Programm erkennt jedoch automatisch, dass es sich um ein Syntaxdiagramm handelt. \\

Hinweis! Speichern sie die Diagramme in einem eindeutigen Ordner oder geben Sie dem Diagramm einen geeigneten Namen, da die Speicherendungen im gesamten \jalgo \emph{"`*.jalgo"'} sind.

\subsubsection{Laden von Syntaxdiagrammen}

Beim Laden von Syntaxdiagrammen unterscheidet das Modul, ob sie vollst�ndig sind oder nicht. Vollst�ndig bedeutet, dass zu jeder Variable im Diagramm ein Diagramm mit diesem Namen existiert. Ist es vollst�ndig, befinden Sie sich nach dem Laden in der Syntaxdiagrammanzeige. Ist es nicht vollst�ndig wird der Editor ge�ffnet und das Diagramm kann bearbeitet werden. 

\newpage