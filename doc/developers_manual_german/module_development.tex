%In Entwicklerhandbuch einzuf�gen

\documentclass[12pt]{article}
\usepackage{a4wide, listings, url}
\usepackage[latin1]{inputenc}
 
\title{J-Algo\\Module Programmers Manual}
\author{Stephan Creutz \and Michael Pradel \and Alexander Claus}
\date{\today}

\newcommand{\code}[1]{\lstinline$#1$}

\begin{document}
\maketitle
\lstset{basicstyle=\small,language=Java,showstringspaces=false}
\section{Introduction to jAlgo}
J-Algo was developed to provide multiple module support. Each module should cover one topic (e.g. tree algorithms or EBNF). For this reason we created a simple interface. This interface is described in the sections below.\\
The software is using the following toolkits: SWT, JFace and Draw2d (see section \ref{seealso}), thus you have to use it too. But it might be possible to write an adapter which makes it possible to use e.g. Swing or something else.

\section{Implementing a module}

To implement a new J-Algo module, create a directory with your modules name in\\ /source/org/jalgo/module. This directory has to contain at least two classes:\\ ModuleConnector.java implementing the interface IModuleConnector and ModuleInfo.java implementing the interface IModuleInfo.\\
In order to help you to write a new module, there is an minimalistic module called\\
testModule. It is a correctly implemented J-Algo module, but does nothing. You can use it as a skeleton for any new module.

\subsection{IModuleConnector}
This interface establishes a connection to the main program.
You have to implement the methods listed below:
\newpage

\begin{lstlisting}[frame=single,caption={IModuleConnector}]
public interface IModuleConnector {

	/**
	 * This method is invoked, after the user loaded a saved file
	 * for the module.
	 *
	 * @param data the loaded file consisting of the module
	 *             header, which was added by the main program
	 *             before saving (e.g. including with which
	 *             module the file is associated) and the data
	 *             for the module; put the data in here
	 */
	public void setDataFromFile(ByteArrayInputStream data);

	/**
	 * This method is invoked, when the user wants to save the
	 * state of the module.
	 *
	 * @return a stream with the data from the module, that has
	 *         to be stored in a file after the main program
	 *         added the module header (e.g. including with
	 *         which module the file is associated) to it
	 */
	public ByteArrayOutputStream getDataForFile();

	/**
	 * This method will be invoked, if the user clicked the
	 * print-button (or chose to print in any other way)
	 * The module will call a print dialog and manage the
	 * printing.
	 *
	 * NOTE! Printing is currently not supported by the J-Algo
	 * main program.
	 */
	public void print();

	/**
	 * Get the Menu from the module
	 */
	public SubMenuManager getMenuManager();




	/**
	 * Get the ToolBar from the module
	 */
	public SubToolBarManager getToolBarManager();

	/**
	 * Get the StatusLine from the module
	 */
	public SubStatusLineManager getStatusLineManager();
	
	/**
	 * Get a class with all module information (name,
	 * description, version, ...)
	 */
	public IModuleInfo getModuleInfo();

	/**
	 * This method is invoked, when module or program are
	 * intended to be closed.
	 * Here the user can be asked for saving his work.
	 * If this method returns false, the closing of module/
	 * program is ignored.
	 * 
	 * @return true, if module is ready to be closed,
	 *         false otherwise
	 */
	public boolean close();
}
\end{lstlisting}

\subsection{ModuleConnector from testModule}
Here is a concrete example for a class implementing the \code{IModuleConnector} interface.

% TODO: wie wird print() umgesetzt, momentan wird dies am interface vorbei gemacht
\begin{lstlisting}[frame=single,caption={IModuleInfo interface}]
public class ModuleConnector implements IModuleConnector {

    private ModuleInfo moduleInfo;

    private ApplicationWindow appWin;
    private Composite comp;
    private SubMenuManager menuManager;
    private SubToolBarManager toolBarManager;
    private SubStatusLineManager statusLineManager;
    

    public ModuleConnector(
		ApplicationWindow appWin,
		Composite comp,
		SubMenuManager menu,
		SubToolBarManager tb,
		SubStatusLineManager sl) {
	    
		moduleInfo = new ModuleInfo();
	    
		this.appWin = appWin;
		this.comp = comp;
		this.menuManager = menu;
		this.toolBarManager = tb;
		this.statusLineManager = sl;
    }
    
    public void run() {
        System.err.println("testModule is running");
    }

    public void setDataFromFile(ByteArrayInputStream data) {
    }

    public ByteArrayOutputStream getDataForFile() {
        return null;
    }

    public void print() {}

    public SubMenuManager getMenuManager() {
        return menuManager;
    }

    public SubToolBarManager getToolBarManager() {
        return toolBarManager;
    }

    public SubStatusLineManager getStatusLineManager() {
        return statusLineManager;
    }

    public IModuleInfo getModuleInfo() {
        return moduleInfo;
    }

    public boolean close() {
        return true;
    }
}
\end{lstlisting}

\subsection{IModuleInfo}
The class which implements this interface provides some basic information about the module. The class yields the name, version, author(s), license, description and a logo. Furthermore it holds the information about open files.

\begin{lstlisting}[frame=single,caption={IModuleInfo interface}]
import org.eclipse.jface.resource.ImageDescriptor

public interface IModuleInfo {
	public String getName();
	public String getVersion();
	public String getAuthor();
	public String getDescription();
	public ImageDescriptor getLogo();
	public String getLicense();
	
	/**
	 * Get the filename of the currently opened file.
	 * @return filename
	 */
	public String getOpenFileName();
	/**
	 * Set the filename of the currently opened file.
	 * @param string filename
	 */
	public void setOpenFileName(String string);
}
\end{lstlisting}

\subsection{ModuleInfo from testModule}
Here is a concrete example for a class implementing the \code{IModuleInfo} interface.

\begin{lstlisting}[frame=single,caption={Sample ModuleInfo Code}]
public class ModuleInfo implements IModuleInfo {

    public String getName() {
        return "testModule";
    }

    public String getVersion() {
        return "0.1";
    }

    public String getAuthor() {
        return "Your Name";
    }

    public String getDescription() {
        return "a module for testing purposes";
    }

    public ImageDescriptor getLogo() {
        return ImageDescriptor.createFromFile(null, "pix/new.gif");
    }

    public String getLicense() {
        return "GPL";
    }

    public String getOpenFileName() {
        return null;
    }

    public void setOpenFileName(String string) {
    }
}
\end{lstlisting}

If this is too easy for you, just have a look at the existing modules, namely "AVL-trees", "Dijkstra-algorithm", "Syntax diagrams and EBNF".
\newpage

\section{Bind the module to the main program}
Binding a new module to the main program is really easy: You only have to add two lines into the class /src/org/jalgo/main/JalgoMain.java.\\
You should find the method addKnownModules:
\begin{lstlisting}[frame=single,caption={JalgoMain.addKnownModules()}]
public void addKnownModules() {
    try {
    	knownModules.add(ModuleConnector.class);
    	knownModules.add(
        	org.jalgo.module.testModule.ModuleConnector.class);
    	//Add a new ModuleConnector here!!
    } catch (Exception e) {
    	e.printStackTrace();
    }
    knownModuleInfos.add(new ModuleInfo());
    knownModuleInfos.add(
    	new org.jalgo.module.testModule.ModuleInfo());
    //Add a new ModuleInfo here!!
}
\end{lstlisting}

Please use the complete path to your classes!

\section{Known bugs}
\begin{itemize}
\item The method \code{public void print()} is unused and senseless in the moment.
\end{itemize}

\section{Reporting bugs}
If you find a bug please be so kind to drop us an email (\url{j-algo-development@lists.sourceforge.net}).

\section{See also}\label{seealso}
To program a module you need information about SWT, JFace and Draw2d. This can be found at \url{http://www.eclipse.org}.\\
More detailed information can be found in the source code.

\end{document}
