\documentclass{scrartcl}
\usepackage[ngerman]{babel}
\usepackage[T1]{fontenc}
\usepackage[ansinew]{inputenc}
\usepackage{a4wide}

%verwenden von grafiken
\usepackage[dvipdf, final]{graphicx}

%verwenden von hyperlinks, stil derselben
\usepackage{color}
\definecolor{darkblue}{rgb}{0,0,0.5}

\usepackage{hyperref}
\hypersetup{
%	draft,										%hyperlinks ausschalten				
	colorlinks,									%hyperlinks farbig darstellen
	linkcolor   = darkblue,
	filecolor   = darkblue,
	urlcolor    = darkblue,
	citecolor   = darkblue,
	pdftitle    = {Entwicklerhandbuch},		%titel
	pdfsubject  = {j-Algo},			%thema
	pdfauthor   = {Alexander Claus, Matthias Schmidt},
	pdfkeywords = {Algorithmen, Visualisierung},
	pdfcreator  = {Distiller},
	pdfproducer = {LaTeX mit Hyperref-Package}}

%verwenden von code-ausschnitten
\usepackage{listings}

%spezielle kommandos
%schreibweise des software-titels
\newcommand{\jalgo}{\mbox{\bfseries {\color{red}j}-Algo} }
%pfad zu den bildern
\newcommand{\pfad}{pics/}
%f�gt ein bild an einer bestimmten stelle, relativ zur position des befehls, ein.
%usage: \icon{dateiname}{y-offset}{x-offset}{bildvergr��erung}
\newcommand{\icon}[4]{
	\vspace{#2 ex}
	\hspace{#3 ex}
	\includegraphics[scale=#4]{\pfad #1}
}
%f�gt ein bild mittig mit bildunterschrift ein.
%usage: \centerpic{dateiname}{bildvergr��erung}{untertitel}
\newcommand{\centerpic}[3]{
	\begin{center}
		\includegraphics[scale=#2]{\pfad #1}\\
		{\small #3}
	\end{center}
}
%f�gt eine \subsection mit einem f�hrenden icon ein
%usage: \subsectionicon{text}{icon}
\newcommand{\subsectionicon}[2]{
	\subsection[#1]{\qquad #1}
	\icon{#2}{-5}{7}{1}
	\\
}
%f�gt eine \subsection mit 2 f�hrenden icons ein
%usage: \subsectiondoubleicon{text}{icon1}{icon2}
\newcommand{\subsectiondoubleicon}[3]{
	\subsection[#1]{\qquad \quad #1}
	\icon{#2}{-5}{7}{1}
	\icon{#3}{0}{-2}{1}
	\\
}
%f�gt eine \subsubsection* mit 2 f�hrenden icons ein
%usage: \subsubsectiondoubleicon{text}{icon1}{icon2}
\newcommand{\subsubsectiondoubleicon}[3]{
	\subsubsection*{\qquad \qquad #1}
	\icon{#2}{-4}{1}{1}
	\icon{#3}{0}{-2}{1}
	\\
}

\begin{document}

\begin{titlepage}
\centerpic{title}{1}{}
\vfill
\begin{flushright}
{\Huge \textbf{Entwicklerhandbuch}}
\end{flushright}
\end{titlepage}

\newpage

\tableofcontents
\newpage

\chapter{Das Modul Dijkstra}
\section{Einleitung}
Das Modul \dijkstra visualisiert den bekannten Algorithmus von E. W. Dijkstra zum Finden der k�rzesten Wege von einem Startknoten in einem Distanzgraphen. Der Algorithmus selbst ist unter anderem im Vorlesungsskript von Prof. Vogler "`Algorithmen, Datenstrukturen und Programmierung"' zu finden. Aber auch im Internet existieren zahlreiche Quellen dazu.

Soweit es m�glich gewesen ist, wurde beim Design des Moduls darauf geachtet, es weitgehend intuitiv und selbst-dokumentierend zu gestalten. Nichtsdestotrotz findet sich hier eine kurze
Einf�hrung in das \dijkstra - Modul.

\section{Funktions�bersicht}
Das Modul \dijkstra realisiert folgende Funktionen:
\begin{itemize}
	\item graphisches Erstellen / Bearbeiten eines Distanzgraphen
	\item Erstellen / Bearbeiten eines Graphen mittels Kanten- / Knotenliste oder Adjazenzmatrix
	\item Speichern und Laden von Graphen
	\item Visualisierung des Dijkstra-Algorithmus
\end{itemize}

\section{Modul starten}
Um das Modul zu starten, w�hlt man im Men� \textsc{<Datei>} das Submen� \textsc{<Neu>} und dann den Men�befehl \textsc{\dijkstra}. Im Hauptfenster erscheint nun die Oberfl�che des \dijkstra - Moduls im Eingabe-Modus.

\section{Symbolleiste}
Die Symbolleiste stellt die Funktionen \textsc{Speichern, Speichern unter, R�ckg�ngig} und \textsc{Wiederherstellen} bereit.
\newpage
\section{CVS-Zugang}
\jalgo ist als Projekt bei \href{https://sourceforge.net/projects/j-algo/}{SourceForge} registriert und gehostet. Es gibt 2 Arten, auf das CVS-Repository des Projektes zuzugreifen.
\begin{enumerate}
	\item Lesezugriff. Als Beobachter des Projektes kann jeder auf das Projekt lesend zugreifen. Die Zugangsdaten sind:\\
		Verbindungsmethode: \verb|pserver|\\
		Host: \verb|cvs.sourceforge.net|\\
		Repository-Pfad: \verb|/cvsroot/j-algo|\\
		Login: \verb|anonymous|
	\item Vollzugriff. Als registrierter Entwickler bei SourceForge und als eingetragenes Projektmitglied bei \jalgo kann das CVS unter folgenden Zugangsdaten im Vollzugriff erreicht werden:\\
		Verbindungsmethode: \verb|extssh|\\
		Host: \verb|cvs.sourceforge.net|\\
		Repository-Pfad: \verb|/cvsroot/j-algo|\\
		Login: <\textsc{SourceForge-Login}>\\
		Passwort: <\textsc{SourceForge-Passwort}>\\
		Um als Projektmitglied eingetragen zu werden, wenden Sie sich bitte an den Projekt-Administrator. Dessen Kontaktdaten sind auf der SourceForge-Seite zug�nglich.
\end{enumerate}
\textbf{Achtung:} Wird das Projekt im Rahmen des Software-Praktikums an der TUD weiterentwickelt, gelten andere Bedingungen f�r den CVS-Zugang. Diese sind beim zust�ndigen Betreuer des Praktikums zu erfragen.

\newpage

\section{Entwickeln unter Eclipse}
Nat�rlich steht es jedem Entwickler frei, eine Programmierumgebung seiner Wahl zu benutzen. Da jedoch der Gro�teil der \jalgo-Entwickler unter \href{http://eclipse.org}{Eclipse} programmiert, und diese Plattform einige komfortable Features besitzt, sollen hier die wichtigsten Einstellungen f�r diese Umgebung erl�utert werden. F�r andere Programmierumgebungen gelten sie sinngem��.\\
Da \jalgo die Java-Version 1.5 verwendet, ist eine Eclipse-Version 3.1 oder h�her erforderlich.\footnote{Achtung, auf der Seite des Eclipse-Projektes wird noch auf die Version 1.4.2 des JDK verlinkt}\\

Das Projekt kann in der CVS-Ansicht von Eclipse ausgecheckt werden. Ab jetzt sind zwar alle n�tigen Daten (Quellcodes, etc.) auf dem Rechner. Allerdings m�ssen noch einige Einstellungen vorgenommen werden, damit das Projekt kompiliert und gestartet werden kann:
\begin{itemize}
	\item Unter den Projekteinstellungen->"'Java Compiler"'->"`Compiler Compliance Level"' ist "`5.0"' einzustellen. Es ist sicherzustellen, dass unter Projekteinstellungen->Libraries als "`JRE System Library"' die Version 1.5 eingestellt ist. Ist dies nicht der Fall, so ist diese mittels "`Add Library..."'->"`JRE System Library"' einzustellen.
	\item Unter Projekteinstellungen->Info->Text file encoding muss UTF-8 eingestellt werden. Dies garantiert reibungslose Unterst�tzung von Umlauten auf verschiedenen Betriebssystemen.
	\item Unter Projekteinstellungen->Java Build Path m�ssen jetzt einige Einstellungen f�r den ClassPath des Projektes vorgenommen werden:\\
Unter Source darf nur der Ordner \textsc{<Projektordner>}\verb|/src| stehen. Ist dies nicht der Fall, ist dieser mittels "`Add Folder..."' aus der Ordnerliste auszuw�hlen. Andere Ordner sind zu entfernen.
	\item Als "`Default Output Folder"' ist \textsc{<Projektordner>}\verb|/bin| anzugeben.
	\item Unter "`Libraries"'->"`Add JARs..."' ist \textsc{<Projektordner>}\verb|/extlibs/jh.jar| hinzuzuf�gen. Dies ist die n�tige Bibliothek f�r das Hilfe-System.
	\item Unter "`Libraries"'->"`Add Class Folder..."' sind folgende Ordner hinzuzuf�gen:\\
		\textsc{<Projektordner>}\verb|/runtime| (f�r die Erkennung der installierten Module)\\
		\textsc{<Projektordner>}\verb|/res/main| (f�r die Ressourcen zum Hauptprogramm)\\
		sowie alle verf�gbaren Modulordner unter
		\textsc{<Projektordner>}\verb|/res/module/|, also z.B.
		\textsc{<Projektordner>}\verb|/res/module/testModule| (f�r die Ressourcen der einzelnen Module)
	\item Als n�chstes werden die jUnit-Bibliotheken ben�tigt. Weil Eclipse diese bereits eingebaut hat, ist die einfachste Variante, diese hinzuzuf�gen, folgenderma�en:\\
Projekteinstellungen �bernehmen, Workspace neu kompilieren lassen, und dann im View "`Problems"' einen der vielen Fehler ausw�hlen, die im Zusammenhang mit jUnit gebracht werden. Beim �ffnen des gew�hlten Source-Files zeigt Eclipse im Editor bei den entsprechenden Imports Fehler an. Dr�cken Sie genau dort auf das rote Kreuz, und Ihnen wird die Option angeboten "`Add jUnit libraries"'. W�hlen Sie diese aus, schlie�en das Source-File, und fertig.\\
Jetzt sollte im View "`Problems"' kein Fehler mehr angezeigt werden.
	\item Nun muss noch eine Startkonfiguration erstellt werden, und dann sind wir fertig:\\
Unter dem Men�punkt Run->Run... erstellen Sie eine neue Konfiguration vom Typ "`Java Application"', vergeben einen sinnvollen Namen und w�hlen vom \jalgo-Projekt als "`Main-Class"' \verb|org.jalgo.main.JAlgoMain| aus.
\end{itemize}

Jetzt ist das Projekt kompilierbar und das Programm kann gestartet werden.

\section{Projektstruktur}
Es folgt ein kurzer �berblick �ber die bestehende Struktur des Projektes, so dass der Entwickler wei�, welche Teile er ver�ndern darf, und welche besser unangetastet bleiben sollten\dots\\
Das Projekt fasst mehrere Ordner und einige "`lose"' Dateien. Der Reihe nach:
\begin{itemize}
	\item Der Ordner \verb|bin| fasst die kompilierten Klassen. Sein Inhalt kann gel�scht werden, er wird bei jedem kompilieren neu erstellt. (Hinweis: Dieser Ordner geh�rt nicht unter die Versionskontrolle!)
	\item Der Ordner \verb|doc| fasst die Projektdokumentation. Dies sind die Dateien zum Entwicklerhandbuch, zum Benutzerhandbuch, sowie einige Dateien, die gewisse aufgetretene Probleme und evtl. Abhilfen schildern.
	\item Im Ordner \verb|examples| sind Beispieldateien f�r jedes Modul enthalten. Der komplette Ordner wird sp�ter in der Distribution enthalten sein.
	\item Im Ordner \verb|extlibs| liegen Bibliotheken, die Fremdcode enthalten. Dies ist derzeit nur die Laufzeitbibliothek des Hilfesystems. Der komplette Ordner wird sp�ter in der Distribution enthalten sein.
	\item Der Ordner \verb|relicts| fasst Codeteile und Ressourcendateien, welche derzeit nicht mehr verwendet werden. Sie wurden trotzdem aufgehoben, weil sie teilweise Funktionalit�t enthalten, die zu implementieren mal begonnen wurde, die jedoch nie ausgereift waren und daher derzeit nicht verwendet werden. Vielleicht bringen Sie einen Nutzen, wenn der Entwickler Ideen sucht.
	\item Im Ordner \verb|res| liegen alle Ressourcendateien geordnet nach Programmteilen.
	\item Der Ordner \verb|runtime| enth�lt leere, aber notwendige Dateien f�r die Laufzeit. Sie sind Teil der Pluginstruktur, und erm�glichen das Erkennen der installierten Module.
	\item Im Ordner \verb|src| schlie�lich ist der Quellcode enthalten. Die Paketstruktur ist intuitiv gehalten. Unter \verb|org.jalgo.main| findet sich alles, was zum Hauptprogramm geh�rt, und unter \verb|org.jalgo.module| liegen alle Modulpakete.
	\item Die "`losen"' Dateien sind diverse Build-Skripte, Manifest- und Start-Dateien f�r verschiedene Betriebssysteme sowie einige projektspezifische Dateien.
\end{itemize}
Teilweise wird in diesem Entwicklerhandbuch auf die API-Dokumentation von \jalgo verwiesen. Da an der Software permanent gearbeitet wird, ist diese nicht unter der Versionskontrolle verf�gbar, sondern sollte vom Entwickler selbst in regelm��igen Abst�nden generiert werden. Arbeitet der Entwickler unter Eclipse, so kann dies unter dem Men�punkt Project->"`Generate Javadoc..."' ganz einfach durchgef�hrt werden.
\newpage
%In Entwicklerhandbuch einzuf�gen

\documentclass[12pt]{article}
\usepackage{a4wide, listings, url}
\usepackage[latin1]{inputenc}
 
\title{J-Algo\\Module Programmers Manual}
\author{Stephan Creutz \and Michael Pradel \and Alexander Claus}
\date{\today}

\newcommand{\code}[1]{\lstinline$#1$}

\begin{document}
\maketitle
\lstset{basicstyle=\small,language=Java,showstringspaces=false}
\section{Introduction to jAlgo}
J-Algo was developed to provide multiple module support. Each module should cover one topic (e.g. tree algorithms or EBNF). For this reason we created a simple interface. This interface is described in the sections below.\\
The software is using the following toolkits: SWT, JFace and Draw2d (see section \ref{seealso}), thus you have to use it too. But it might be possible to write an adapter which makes it possible to use e.g. Swing or something else.

\section{Implementing a module}

To implement a new J-Algo module, create a directory with your modules name in\\ /source/org/jalgo/module. This directory has to contain at least two classes:\\ ModuleConnector.java implementing the interface IModuleConnector and ModuleInfo.java implementing the interface IModuleInfo.\\
In order to help you to write a new module, there is an minimalistic module called\\
testModule. It is a correctly implemented J-Algo module, but does nothing. You can use it as a skeleton for any new module.

\subsection{IModuleConnector}
This interface establishes a connection to the main program.
You have to implement the methods listed below:
\newpage

\begin{lstlisting}[frame=single,caption={IModuleConnector}]
public interface IModuleConnector {

	/**
	 * This method is invoked, after the user loaded a saved file
	 * for the module.
	 *
	 * @param data the loaded file consisting of the module
	 *             header, which was added by the main program
	 *             before saving (e.g. including with which
	 *             module the file is associated) and the data
	 *             for the module; put the data in here
	 */
	public void setDataFromFile(ByteArrayInputStream data);

	/**
	 * This method is invoked, when the user wants to save the
	 * state of the module.
	 *
	 * @return a stream with the data from the module, that has
	 *         to be stored in a file after the main program
	 *         added the module header (e.g. including with
	 *         which module the file is associated) to it
	 */
	public ByteArrayOutputStream getDataForFile();

	/**
	 * This method will be invoked, if the user clicked the
	 * print-button (or chose to print in any other way)
	 * The module will call a print dialog and manage the
	 * printing.
	 *
	 * NOTE! Printing is currently not supported by the J-Algo
	 * main program.
	 */
	public void print();

	/**
	 * Get the Menu from the module
	 */
	public SubMenuManager getMenuManager();




	/**
	 * Get the ToolBar from the module
	 */
	public SubToolBarManager getToolBarManager();

	/**
	 * Get the StatusLine from the module
	 */
	public SubStatusLineManager getStatusLineManager();
	
	/**
	 * Get a class with all module information (name,
	 * description, version, ...)
	 */
	public IModuleInfo getModuleInfo();

	/**
	 * This method is invoked, when module or program are
	 * intended to be closed.
	 * Here the user can be asked for saving his work.
	 * If this method returns false, the closing of module/
	 * program is ignored.
	 * 
	 * @return true, if module is ready to be closed,
	 *         false otherwise
	 */
	public boolean close();
}
\end{lstlisting}

\subsection{ModuleConnector from testModule}
Here is a concrete example for a class implementing the \code{IModuleConnector} interface.

% TODO: wie wird print() umgesetzt, momentan wird dies am interface vorbei gemacht
\begin{lstlisting}[frame=single,caption={IModuleInfo interface}]
public class ModuleConnector implements IModuleConnector {

    private ModuleInfo moduleInfo;

    private ApplicationWindow appWin;
    private Composite comp;
    private SubMenuManager menuManager;
    private SubToolBarManager toolBarManager;
    private SubStatusLineManager statusLineManager;
    

    public ModuleConnector(
		ApplicationWindow appWin,
		Composite comp,
		SubMenuManager menu,
		SubToolBarManager tb,
		SubStatusLineManager sl) {
	    
		moduleInfo = new ModuleInfo();
	    
		this.appWin = appWin;
		this.comp = comp;
		this.menuManager = menu;
		this.toolBarManager = tb;
		this.statusLineManager = sl;
    }
    
    public void run() {
        System.err.println("testModule is running");
    }

    public void setDataFromFile(ByteArrayInputStream data) {
    }

    public ByteArrayOutputStream getDataForFile() {
        return null;
    }

    public void print() {}

    public SubMenuManager getMenuManager() {
        return menuManager;
    }

    public SubToolBarManager getToolBarManager() {
        return toolBarManager;
    }

    public SubStatusLineManager getStatusLineManager() {
        return statusLineManager;
    }

    public IModuleInfo getModuleInfo() {
        return moduleInfo;
    }

    public boolean close() {
        return true;
    }
}
\end{lstlisting}

\subsection{IModuleInfo}
The class which implements this interface provides some basic information about the module. The class yields the name, version, author(s), license, description and a logo. Furthermore it holds the information about open files.

\begin{lstlisting}[frame=single,caption={IModuleInfo interface}]
import org.eclipse.jface.resource.ImageDescriptor

public interface IModuleInfo {
	public String getName();
	public String getVersion();
	public String getAuthor();
	public String getDescription();
	public ImageDescriptor getLogo();
	public String getLicense();
	
	/**
	 * Get the filename of the currently opened file.
	 * @return filename
	 */
	public String getOpenFileName();
	/**
	 * Set the filename of the currently opened file.
	 * @param string filename
	 */
	public void setOpenFileName(String string);
}
\end{lstlisting}

\subsection{ModuleInfo from testModule}
Here is a concrete example for a class implementing the \code{IModuleInfo} interface.

\begin{lstlisting}[frame=single,caption={Sample ModuleInfo Code}]
public class ModuleInfo implements IModuleInfo {

    public String getName() {
        return "testModule";
    }

    public String getVersion() {
        return "0.1";
    }

    public String getAuthor() {
        return "Your Name";
    }

    public String getDescription() {
        return "a module for testing purposes";
    }

    public ImageDescriptor getLogo() {
        return ImageDescriptor.createFromFile(null, "pix/new.gif");
    }

    public String getLicense() {
        return "GPL";
    }

    public String getOpenFileName() {
        return null;
    }

    public void setOpenFileName(String string) {
    }
}
\end{lstlisting}

If this is too easy for you, just have a look at the existing modules, namely "AVL-trees", "Dijkstra-algorithm", "Syntax diagrams and EBNF".
\newpage

\section{Bind the module to the main program}
Binding a new module to the main program is really easy: You only have to add two lines into the class /src/org/jalgo/main/JalgoMain.java.\\
You should find the method addKnownModules:
\begin{lstlisting}[frame=single,caption={JalgoMain.addKnownModules()}]
public void addKnownModules() {
    try {
    	knownModules.add(ModuleConnector.class);
    	knownModules.add(
        	org.jalgo.module.testModule.ModuleConnector.class);
    	//Add a new ModuleConnector here!!
    } catch (Exception e) {
    	e.printStackTrace();
    }
    knownModuleInfos.add(new ModuleInfo());
    knownModuleInfos.add(
    	new org.jalgo.module.testModule.ModuleInfo());
    //Add a new ModuleInfo here!!
}
\end{lstlisting}

Please use the complete path to your classes!

\section{Known bugs}
\begin{itemize}
\item The method \code{public void print()} is unused and senseless in the moment.
\end{itemize}

\section{Reporting bugs}
If you find a bug please be so kind to drop us an email (\url{j-algo-development@lists.sourceforge.net}).

\section{See also}\label{seealso}
To program a module you need information about SWT, JFace and Draw2d. This can be found at \url{http://www.eclipse.org}.\\
More detailed information can be found in the source code.

\end{document}

\section{Bekannte Fehler und Schwachstellen}
Derzeit wird vom Hauptprogramm kein Drucken unterst�tzt. Daher ist die Methode\\
\verb|AbstractModuleConnector.print()| derzeit nutzlos. Sie wird trotzdem aus Kompatibilit�tsgr�nden mitgef�hrt, um f�r zuk�nftige Implementationen ger�stet zu sein.

\section{Weiterf�hrende Links}
Die Software nutzt f�r die graphische Oberfl�che die \textbf{Swing}-Technologie. Informationen hierzu findet man unter\\
\url{http://java.sun.com/docs/books/tutorial/},\\
\url{http://java.sun.com/products/jfc/}\\
oder in der API-Dokumentation von Java.

\end{document}