\section{CVS-Zugang}
\jalgo ist als Projekt bei \href{https://sourceforge.net/projects/j-algo/}{SourceForge} registriert und gehostet. Es gibt 2 Arten, auf das CVS-Repository des Projektes zuzugreifen.
\begin{enumerate}
	\item Lesezugriff. Als Beobachter des Projektes kann jeder auf das Projekt lesend zugreifen. Die Zugangsdaten sind:\\
		Verbindungsmethode: \verb|pserver|\\
		Host: \verb|cvs.sourceforge.net|\\
		Repository-Pfad: \verb|/cvsroot/j-algo|\\
		Login: \verb|anonymous|
	\item Vollzugriff. Als registrierter Entwickler bei SourceForge und als eingetragenes Projektmitglied bei \jalgo kann das CVS unter folgenden Zugangsdaten im Vollzugriff erreicht werden:\\
		Verbindungsmethode: \verb|extssh|\\
		Host: \verb|cvs.sourceforge.net|\\
		Repository-Pfad: \verb|/cvsroot/j-algo|\\
		Login: <\textsc{SourceForge-Login}>\\
		Passwort: <\textsc{SourceForge-Passwort}>\\
		Um als Projektmitglied eingetragen zu werden, wenden Sie sich bitte an den Projekt-Administrator. Dessen Kontaktdaten sind auf der SourceForge-Seite zug�nglich.
\end{enumerate}
\textbf{Achtung:} Wird das Projekt im Rahmen des Software-Praktikums an der TUD weiterentwickelt, gelten andere Bedingungen f�r den CVS-Zugang. Diese sind beim zust�ndigen Betreuer des Praktikums zu erfragen.

\newpage

\section{Entwickeln unter Eclipse}
Nat�rlich steht es jedem Entwickler frei, eine Programmierumgebung seiner Wahl zu benutzen. Da jedoch der Gro�teil der \jalgo-Entwickler unter \href{http://eclipse.org}{Eclipse} programmiert, und diese Plattform einige komfortable Features besitzt, sollen hier die wichtigsten Einstellungen f�r diese Umgebung erl�utert werden. F�r andere Programmierumgebungen gelten sie sinngem��.\\
Da \jalgo die Java-Version 1.5 verwendet, ist eine Eclipse-Version 3.1 oder h�her erforderlich.\\

Das Projekt kann in der CVS-Ansicht von Eclipse ausgecheckt werden. Ab jetzt sind zwar alle n�tigen Daten (Quellcodes, etc.) auf dem Rechner. Allerdings m�ssen noch einige Einstellungen vorgenommen werden, damit das Projekt kompiliert und gestartet werden kann:
\begin{itemize}
	\item Unter den Projekteinstellungen->Info->Text file encoding muss UTF-8 eingestellt werden. Dies garantiert reibungslose Unterst�tzung von Umlauten auf verschiedenen Betriebssystemen.
	\item Unter Projekteinstellungen->Java Build Path m�ssen jetzt einige Einstellungen f�r den ClassPath des Projektes vorgenommen werden:\\
Unter Source darf nur der Ordner \textsc{<Projektordner>}\verb|/src| stehen. Andere Ordner sind zu entfernen.
	\item Als "`Default Output Folder"' ist \textsc{<Projektordner>}\verb|/bin| anzugeben.
	\item Unter "`Libraries"'->"`Add JARs..."' ist \textsc{<Projektordner>}\verb|/extlibs/jh.jar| hinzuzuf�gen. Dies ist die n�tige Bibliothek f�r das Hilfe-System.
	\item Unter "`Libraries"'->"`Add Class Folder..."' sind folgende Ordner hinzuzuf�gen:\\
		\textsc{<Projektordner>}\verb|/runtime| (f�r die Erkennung der installierten Module)\\
		\textsc{<Projektordner>}\verb|/res/main| (f�r die Ressourcen zum Hauptprogramm)\\
		sowie alle verf�gbaren Modulordner unter
		\textsc{<Projektordner>}\verb|/res/modules/|, also z.B.
		\textsc{<Projektordner>}\verb|/res/modules/testModule| (f�r die Ressourcen der einzelnen Module)
	\item Als n�chstes werden die jUnit-Bibliotheken ben�tigt. Weil Eclipse diese bereits eingebaut hat, ist die einfachste Variante, diese hinzuzuf�gen, folgenderma�en:\\
Projekteinstellungen schliessen, Workspace neu kompilieren lassen, und dann im View "`Problems"' einen der vielen Fehler ausw�hlen, die im Zusammenhang mit jUnit gebracht werden. Beim �ffnen des gew�hlten Source-Files zeigt Eclipse im Editor bei den entsprechenden Imports Fehler an. Dr�cken Sie genau dort auf das rote Kreuz, und Ihnen wird die Option angeboten "`Add jUnit libraries"'. W�hlen Sie diese aus, schlie�en das Source-File, und fertig.\\
Jetzt sollte im View "`Problems"' kein Fehler mehr angezeigt werden.
	\item Nun muss noch eine Startkonfiguration erstellt werden, und dann sind wir fertig:\\
Unter dem Men�punkt Run->Run... erstellen Sie eine neue Konfiguration vom Typ "`Java Application"', vergeben einen sinnvollen Namen und w�hlen vom \jalgo-Projekt als "`Main-Class"' \verb|org.jalgo.main.JAlgoMain| aus.
\end{itemize}

Jetzt ist das Projekt kompilierbar und das Programm kann gestartet werden.

\section{Projektstruktur}
Es folgt ein kurzer �berblick �ber die bestehende Struktur des Projektes, so dass der Entwickler wei�, welche Teile er ver�ndern darf, und welche er besser unangetastet lassen sollte \dots

Das Projekt fasst mehrere Ordner und einige "`lose"' Dateien. Der Reihe nach:\\
Der Ordner \verb|bin| fasst die kompilierten Klassen. Sein Inhalt kann gel�scht werden, er wird bei jedem kompilieren neu erstellt. (Hinweis: Dieser Ordner geh�rt nicht unter die Versionskontrolle!)\\
Der Ordner \verb|doc| fasst die Projektdokumentation. Dies sind die Dateien zum Entwicklerhandbuch, zum Benutzerhandbuch, sowie einige Dateien, die gewisse aufgetretene Probleme und evtl. Abhilfen schildern.\\
Im Ordner \verb|examples| sind Beispieldateien f�r jedes Modul enthalten. Der komplette Ordner wird sp�ter in der Distribution enthalten sein.\\
Im Ordner \verb|extlibs| liegen Bibliotheken, die Fremdcode enthalten. Dies ist derzeit nur die Laufzeitbibliothek des Hilfesystems. Der komplette Ordner wird sp�ter in der Distribution enthalten sein.\\
Der Ordner \verb|relicts| fasst Codeteile und Ressourcendateien, welche derzeit nicht mehr verwendet werden. Sie wurden trotzdem aufgehoben, weil sie teilweise Funktionalit�t enthalten, die zu implementieren mal begonnen wurde, die jedoch nie ausgereift waren und daher derzeit nicht verwendet werden. Vielleicht bringen Sie einen Nutzen, wenn der Entwickler Ideen sucht.\\
Im Ordner \verb|res| liegen alle Ressourcendateien geordnet nach Programmteilen.\\
Der Ordner \verb|runtime| enth�lt leere, aber notwendige Dateien f�r die Laufzeit. Sie sind Teil der Pluginstruktur, und erm�glichen das Erkennen der installierten Module.\\
Im Ordner \verb|src| schlie�lich ist der Quellcode enthalten. Die Paketstruktur ist in Abbildung X ersichtlich.\\
TODO: Abbildung Paketstruktur.
