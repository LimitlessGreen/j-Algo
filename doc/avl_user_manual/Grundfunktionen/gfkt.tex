
\section{Grundfunktionen}

Die Grundfunktionen basieren auf den vom Hauptprogramm JAlgo bereitgestellten Funktionen und sind f"ur jedes Modul gleich.
Sie umfassen das Laden und Speichern von Moduldaten, hier B"aumen, und das "Offnen eines neuen Moduls. Die Grundfunktionen sind "uber die Werkzeugleiste oder den Men"upunkt 
{\sc "<Datei">} erreichbar.


\medskip
\subsection[Neues Modul "offnen]{ }
\vspace{-4.3ex} {\bf {\large \qquad \quad \ \ Neues Modul "offnen}} 

\icon{icon-new}{-4}{5.5}{1}

Ein Klick auf den {\sc Neu}-Button gibt Ihnen die M"oglichkeit ein beliebiges neues Modul zu "offnen. Die Auswahl  
beschr"ankt sich dabei auf die installierten Module. \\
\\
{\bf Hinweis} \\
In der momentanen Version des Hauptprogramms JAlgo wird per Klick auf den {\sc Neu}-Button jediglich ein neues {\sc EBNF}-Modul
ge"offnet. \\


\medskip
\subsection[Gespeicherten Baum laden]{}
\vspace{-4.3ex} {\bf {\large \qquad \quad \ \ Gespeicherten Baum laden}} 

\icon{icon-open}{-4}{5.5}{1}

Mit einem Klick auf den {\sc "Offnen}-Button erscheint ein Dialogfenster, in dem Ihnen die M"oglichkeit 
gegeben wird, eine *.jalgo-Datei auszuw"ahlen, in welcher ein Baum gespeichert wurde.  \\
Achtung: Da jedes Modul von JAlgo seine Daten in einer solchen *.jalgo-Datei ablegt, kann man beim Blick auf die unge"offnete Datei nicht erkennen, mit welchem Modul diese assoziiert wurde. Es wird jeweils das assoziierte Modul zu der geladenen Datei ge"offnet. Achten Sie daher bei der Vergabe der Dateinamen auf m"oglichst eindeutige Bezeichner.


\medskip
\subsection[Baum speichern]{}
\vspace{-4.3ex} {\bf {\large \qquad \qquad \quad \ Baum speichern}} 

\icon{icon-saveall}{-4}{5.5}{6.3}

Per Klick auf die Buttons {\sc Speichern} und {\sc Speichern unter} k"onnen Sie einen von Ihnen erstellten Baum in einer 
Datei mit der Endung "`.jalgo"' speichern. Wie schon beim Laden eines Baumes "offnet sich ein entsprechendes Fenster, 
in das Sie Zielpfad und Name der neuen Datei eintragen m"ussen. Die Angabe der Dateiendung ist nicht n"otig, das Programm 
erg"anzt diese automatisch. \\
Die Speicherfunktion steht nur zur Verf"ugung, wenn gerade kein Algorithmus l"auft. Sollte noch ein Algorithmus aktiv sein, 
so beenden Sie diesen bitte vorher oder brechen ihn ab.

\centerpic{loaddialog}{0.45}{Das Dialogfenster zum Speichern eines Baumes}

\medskip
\subsection{Modul schlie"sen}


Sie haben die M"oglichkeit, jede Modulinstanz durch Klick auf das Kreuz der dazugeh"origen Registerkarte zu schlie"sen. 
Dabei werden Sie gegebenenfalls gefragt, ob Sie Ihre Arbeit speichern wollen. 
Um das gesamte Programm zu schlie"sen, ist es nicht n"otig, die Module einzeln zu schliessen, das erledigt das Programm 
f"ur Sie.

\bigskip
\begin{center}
	\raisebox{-7ex}{\includegraphics[scale=0.8]{\pfad closecross}} \hfill
	\includegraphics[scale=0.55]{\pfad closemessage} \\
	{\small Der Knopf zum Schlie"sen eines Moduls.} \hfill
	{\small Die Abfrage, ob der Baum gespeichert werden soll.}
\end{center}
\bigskip
\bigskip