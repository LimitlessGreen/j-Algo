
\section{Algorithmussteuerung}

Aufgabe des Moduls \AVL \ ist es, Baumalgorithmen, wie das Einf"ugen und L"oschen von Knoten, zu visualisieren.
Jeder Algorithmus ist in verschiedene Teilschritte unterteilt, die nacheinander angezeigt werden. Das Visualisieren
erfolgt dabei durch das Zeichnen des Baumes, durch die Erkl"arung der Schritte im Dokumentationsbereich und im Logbuch und durch die 
Neuberechnung der baumspezifischen Daten, die im Infobereich pr"asentiert werden.
\\
Nachdem Sie einen Algorithmus gestartet haben, verweilt er in einem Initialzustand und wartet auf Ihre Eingabe. 
Nun haben Sie die M"oglichkeit, den Algorithmus in kleinen oder gro"sen Schritten zu durchlaufen; Sie k"onnen ihn sofort
beenden oder direkt abbrechen. Daf"ur bietet die Algorithmussteuerung die entsprechenden Werkzeuge. \\


\subsection{Schritt-Pfeile}

Mittels der Schritt-Pfeile steuern Sie die Abfolge der Einzelschritte und bekommen so eine detailierte Sicht
auf die Arbeitsweise des Algorithmus. Das Programm bietet Ihnen die M"oglichkeit, einen Teilschritt r"uckg"angig 
zu machen und damit gewisse Abl"aufe zu wiederholen. Die Schritt-Pfeile, welche die R"uckg"angigfunktion anbieten, weisen
in ihrer Richtung nach links und sind dadurch gut von den Vorw"arts-Pfeilen zu unterscheiden. \\
Zus"atzlich gibt es f"ur jede Richtung einen gro"sen und einen kleinen Schritt, der per Knopfdruck ausgef"uhrt wird. \\

Kleine Schritte beim Einf"ugen eines Knotens stellen Schl"usselvergleiche, Balancenberechnungen und
Rotationen dar. Gro"se Schritte hingegen sind zum Beispiel das Suchen der Einf"ugestelle, das Einf"ugen an dieser und 
die gesamte Balancenaktualisierung. \\

\medskip
\raisebox{-1.3ex}{\includegraphics[scale=6.2]{\pfad icon-steps}} {\bf Einzel-Schritt-Pfeile} \\ 

Ein Klick auf diese Buttons realisiert einen kleinen Algorithmusschritt zur"uck bzw. nach vorn. \\


\medskip
\raisebox{-1.3ex}{\includegraphics[scale=6.2]{\pfad icon-stepsblock}} {\bf Block-Schritt-Pfeile} \\

Ein Klick auf diese Buttons realisiert einen gro"sen Schritt zur"uck bzw. nach vorn. Sollte der Algorithmusablauf an eine
Stelle geraten, an der es nur noch einen kleinen Schritt nach vorn bzw. zur"uck gibt, so hat der Block-Schritt die selbe
Funktionalit"at wie ein Einzel-Schritt. \\


\subsection[Abbruch und Beenden-Buttons]{}
\vspace{-4.3ex} {\bf {\large  \qquad \qquad \quad  Abbruch und Beenden-Buttons}} 

\icon{icon-abort&finish}{-3}{5}{6.2}


Klicken Sie auf den Beenden-Button \raisebox{-1ex}{\includegraphics[scale=0.8]{\pfad icon-finish}} um den laufenden 
Algorithmus bis zum Ende auszuf"uhren. \\
Klicken Sie auf den Abbruch-Button \raisebox{-1ex}{\includegraphics[scale=0.8]{\pfad icon-abort}} um den laufenden 
Algorithmus abzubrechen. Der Baum hat danach den gleichen Status wie vor Begin des Algorithmus. \\
Ist ein Algorithmus beendet, so steht Ihnen diese Option nicht mehr zur Verf"ugung, weil nur der
{\it laufende} Algorithmus abgebrochen werden kann. \\


\subsection{Animationsgeschwindigkeit}

Beim Generieren eines Zufallsbaumes haben Sie die Option, den Ablauf der Baumerzeugung als Animation ablaufen zu lassen.
Starten Sie in diesem Modus, so beginnt die Animation sofort und kann mit dem Geschwindigkeitsregler schneller oder langsamer abgespielt
werden. Zu Beginn steht dieser auf der Mittelposition. Verschieben Sie den Regler nach links, um die Animation zu verlangsamen
bzw. nach rechts, um sie zu beschleunigen. \\
Eine Animation der anderen Algorithmenabl"aufe ist in dieser Version von \AVL \ nicht integriert.

\bigskip
\centerpic{animregler}{0.9}{Der Regler f"ur die Animationsgeschwindigkeit}

\bigskip
\bigskip