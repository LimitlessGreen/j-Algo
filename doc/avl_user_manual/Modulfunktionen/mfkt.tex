
\section{Modulfunktionen}

Alle Funktionen des Moduls \AVL \ lassen sich "uber den Kontroll-Bereich der Arbeitsfl"ache bedienen. Sie stellen die 
verschiedenen Baumalgorithmen dar, deren Visualisierung Aufgabe dieses Moduls ist. Grundlegend l"auft die Arbeit mit den Algorithmen immer
nach dem gleichen Schema ab:
\begin{enumerate}
\item \label{key} Schl"usseleingabe
\item Starten des Algorithmus per Klick auf den entsprechenden Button
\end{enumerate}
Es gibt nat"urlich auch Algorithmen, wie der AVL-Test, die keinen Schl"ussel ben"otigen und ohne Schritt \ref{key} auskommen.
\\ \\
Es folgen nun, die einzelnen Funktionen im Detail. 


\medskip
\subsection{Schl"usseleingabe}
F"ur die Eingabe der Schl"usselwerte steht ein Textfeld und ein Button f"ur zuf"allige Werte zur Verf"ugung. 
Es sind nur ganzzahlige Schl"usselwerte zwischen 1 und 99 erlaubt. \\
"Uber dem Textfeld befindet sich eine Nachrichtenzeile, in welcher Sie auf eventuelle Fehleingaben aufmerksam gemacht werden. 
Hier werden sp"ater ebenfalls kurze Ergebnismeldungen zu den Algorithmen eingeblendet. \\


\subsection{Algorithmusfunktionen}
{\bf Knoten einf"ugen} \\
Der eingegebene Wert wird als Schl"ussel f"ur einen neuen Knoten verwendet, der in den Baum eingef"ugt werden soll. Ist 
bereits ein Knoten mit dem gleichen Schl"ussel im Baum enthalten, so bricht der Algorithmus erfolglos ab. \\ \\

{\bf Knoten suchen} \\
Nach dem Starten dieses Algorithmus beginnt die Suche nach dem eingegebenen Schl"ussel im Baum. \\ \\
{\bf Knoten l"oschen} \\
Nach dem eingegebenen Schl"ussel wird gesucht, und wenn ein entsprechender Knoten gefunden wurde, wird dieser aus der 
Baumstruktur entfernt. \\


\subsection{AVL-Modus}
Ist dieses K"astchen aktiviert, werden die entsprechenden Algorithmen so ausgef"uhrt, dass die AVL-Eigenschaft gewahrt
bleibt. Es ist keine Funktion implementiert, die an einem beliebigen Suchbaum die AVL-Eigenschaft herstellt. 
Ist das K"astchen deaktiviert, ist es daher nicht immer ohne weiteres wieder zu aktivieren. Dazu muss zuerst getestet werden, 
ob der Baum die AVL-Eigenschaft hat. Es ist jedoch jederzeit m"oglich, das K"astchen zu deaktivieren und einen 
unbalancierten Baum zu erzeugen. \\


\subsection{Baum auf AVL-Eigenschaft testen}
Wenn der AVL-Modus einmal deaktiviert sein sollte, so erm"oglicht das Programm einen Test des Baumes auf die AVL-Eigenschaft. 
Dabei erfolgt eine Berechnung und Anzeige der Balancen aller Knoten und das eventuelle Markieren von Knoten, deren Balance 
sich nicht mehr im Rahmen der AVL-Eigenschaft bewegt. \\
Es wird ein Hinweis-Dialog ge"offnet, der Ihnen das Ergebnis des Tests pr"asentiert. Sollte der Baum tats"achlich die 
AVL-Eigenschaft besitzen, so wird Ihnen angeboten, direkt in den AVL-Modus zu wechseln.

\bigskip
\centerpic{avltest}{0.7}{Hinweisfenster des AVL-Tests}


\subsection{Baum l"oschen}

Mit einem Klick auf den Button \raisebox{-1.5ex}{\includegraphics[scale=1]{\pfad icon-clear}} in der Werkzeugleiste k"onnen Sie 
nach einer Sicherheitsabfrage die gesamte Baumstruktur l"oschen und mit einer leeren Arbeitsfl"ache neu beginnen.

\bigskip
\centerpic{cleartreemessage}{0.7}{Sicherheitsabfrage beim L"oschen des Baumes}

