

\section{Dokumentation}

Da das Modul \AVL \ vor allem zu Lehr- und Lernzwecken eingesetzt werden soll, ist eine detailierte Dokumentation der 
Algorithmen unumg"anglich. F"ur die Einzelheiten des Algorithmustextes steht ein Skriptauszug zur Verf"ugung.
Ein Logbuch in der rechten unteren Ecke des Bildschirmes f"uhrt Protokoll "uber den Stand und die Beschaffenheit
des einzelnen Algorithmusteilschrittes. \\
Zu guter Letzt wird ein Infobereich angeboten, in dem wichtige Baumdaten zusammengefasst sind. \\

\subsection{Skript}
Der Dokumentationsbereich, der das Skript enth"alt, befindet sich am unteren Bildschirmrand. Es handelt sich hierbei 
um einen Auszug des Skripts zur Vorlesung "`Algorithmen und Datenstrukturen"' von Prof. Vogler (TU Dresden), 
Version vom 2. Oktober 2003. Im Rahmen dieser Vorlesung soll das Modul vorwiegend eingesetzt werden. \\
Bei dem jeweils aktuellen Algorithmustext handelt es sich um die Aktion, die als n"achstes im Ablauf des Algorithmus
erfolgen wird. Sie wird rot markiert angezeigt. \\

\subsection{Logbuch}
Das Logbuch ist eine weitere M"oglichkeit, den Ablauf des Algorithmus zu verfolgen. Es bezieht sich in erster Linie auf
baumspezifische Daten und verwendet zum Beispiel konkrete Schl"usselwerte, anhand deren die Aktionen des Algorithmus 
besser verstanden werden sollen. \\
Auch hier wird der aktuelle Eintrag rot markiert dargestellt. Dieser bezieht sich aber auf die zuletzt ausgef"uhrte 
Aktion. \\

\newpage

\subsection{Infobereich}
Der Infobereich ist haupts"achlich daf"ur gedacht, Ihnen schnell wichtige Daten des Baumes bereit zu stellen. Hier finden
Sie folgende Punkte:

\begin{itemize}
	\item {\sc Anzahl der Knoten} \\
			Dieser Punkt fast f"ur Sie die Anzahl der Knoten im Baum zusammen.
	\item {\sc B"aumh"ohe} \\
			Hier finden Sie die Anzahl der Level des Baumes.
	\item {\sc Durchschnittliche Suchtiefe} \\
			Dieser Wert berechnet sich, durch die Summe der Level aller Knoten geteilt durch die Anzahl dieser. Der Wert 
			ist ein Indiz daf"ur, wie gut der Baum ausbalanciert ist bzw. wie gro"s der Suchaufwand im Durchschnitt ist.
\end{itemize}

\bigskip
\bigskip