\section{Programmstart - Der Willkommensbildschirm}
Nach Starten des Hauptprogramms \jalgo k�nnen Sie �ber den Button <\textsc{Neu}> oder mit dem Men�punkt \textsc{<Datei>$\rightarrow$<Neu>$\rightarrow$<\avl>} eine neue Instanz des Moduls \avl �ffnen. Anschlie�end �ffnet sich der Willkommensbildschirm des Moduls, der Ihnen verschiedene M�glichkeiten er�ffnet.\\
\bigskip
\centerpic{avl/welcomscreen}{0.5}{Der Willkommensbildschirm des Moduls \avl}
\bigskip

\subsectionicon{Baum laden}{avl/welcome_load}
Mit Klick auf das Ordner-Symbol �ffnet sich ein Dialogfenster, in dem Ihnen die M�glichkeit 
gegeben wird, eine \emph{"`*.jalgo"'} - Datei auszuw�hlen, in welcher ein Baum gespeichert wurde.\\
Im Prinzip ist die Bedeutung dieses Buttons die gleiche wie des <\textsc{�ffnen}>-Buttons in der Werkzeugleiste. Der Unterschied besteht darin, dass der Button in der Werkzeugleiste eine neue Modulinstanz �ffnet, in welcher die Datei geladen wird, der Button im Startbildschirm von \avl jedoch die Datei in die aktuell ge�ffnete Modulinstanz l�dt.

\subsectionicon{Baum von Hand erstellen}{avl/welcome_manual}
Mit Klick auf das Hand-Symbol gelangen Sie sofort zur leeren Arbeitsfl�che des Moduls \avl.
Sie k�nnen jetzt mit der knotenweisen Generierung eines neuen Suchbaumes beginnen.

\subsectionicon{Zufallsbaum erstellen lassen}{avl/welcome_random}
Mit Klick auf das W�rfel-Symbol beginnen Sie die Generierung eines zuf�llig erzeugten Suchbaumes. In dem folgenden Dialogfenster k�nnen Sie verschiedene Daten zum Baum und die Art der Visualisierung festlegen. 
\centerpic{avl/rgd}{1}{Eingabe der Zufallsbaumdaten}
\begin{itemize}
	\item {\bf Anzahl der Knoten}\\
	 Geben Sie hier die Anzahl der Knoten ein. Der entstehende Baum muss mindestens einen Knoten enthalten, h�chstens aber 99. 
	\item {\bf AVL-Eigenschaft}\\
	 Aktivieren Sie dieses K�stchen, wenn der zu erstellende Baum die AVL-Eigenschaft besitzen soll.
	\item {\bf Visualisierung} W�hlen Sie hier die Art der Visualisierung der Erstellung aus.
	\begin{itemize}
		\item {\sc keine}\\ Der Baum wird sofort erstellt.
		\item {\sc schrittweise}\\ Jeder Algorithmusschritt kann von Ihnen per Hand best�tigt werden.
		\item {\sc automatisch}\\ Lassen Sie die Erstellung des Baumes als Animation ablaufen, die Geschwindigkeit ist dabei einstellbar. 
	\end{itemize}
	Haben Sie schrittweise oder automatische Visualisierung gew�hlt, k�nnen Sie den Ablauf jederzeit abbrechen. 
    Dabei wird das gerade aktive Knoteneinf�gen abgebrochen, und der Baum steht mit entsprechend weniger Knoten zur Verf�gung.
\end{itemize}

\subsectionicon{Willkommensbildschirm anzeigen}{avl/logo}
Mit Klick auf diesen Button in der Werkzeugleiste des Modulbildschirms kann der Willkommensbildschirm sp�ter jederzeit wieder angezeigt werden. Dabei werden Sie eventuell gefragt, wie Sie mit Ihren �nderungen verfahren wollen. Sollten Sie Ihre �nderungen nicht verwerfen wollen, so wird eine neue Instanz des Moduls ge�ffnet.
\centerpic{avl/clearmessage}{1}{Dialog mit der Frage, ob der ganze Baum gel�scht werden soll.}

\bigskip