
\section{Zusatzfunktionen und Hilfe}
Dieses Kapitel widmet sich den Eastereggs und der Hilfe des Moduls \AVL . \\
Sollten Sie sich lieber selber gerne auf die Suche nach diesen Zusatzfunktionen machen wollen, so "uberspringen Sie besser
dieses Kapitel. \\
F"ur alle Anderen folgt nun eine "Ubersicht zum Baumnavigator, dem Beamermodus und der Hilfefunktion. \\


\subsection{Navigator}
Der Navigator ist eine kleine, versteckte Zusatzfunktion, die die Arbeit mit gro"sen B"aumen erheblich vereinfachen kann.
Er stellt eine willkommene Hilfe f"ur das Scrollen der Zeichenfl"ache dar, ist aber nicht so einfach zu finden. \\
\begin{itemize}
	\item Wenn der Baum, der auf der Zeichenfl"ache angezeigt wird, zu gro"s f"ur diese wird, so erscheinen Scrollbars, 
			mit denen Sie den Bildausschnitt verschieben k"onnen.
	\item Klicken Sie nun auf das kleine Quadrat in der rechten unteren Ecke der Zeichenfl"ache, genau zwischen den beiden
			Scrollbars. Halten Sie dabei die Maustaste gedr"uckt.
	\item Eine kleine "Ubersichtskarte des Baumes mit einem Ausschnittfenster erscheint. Bewegen Sie die Maus (mit 
			gedr"uckter Taste) und das Ausschnittfenster, das den Bildschirminhalt der Zeichenfl"ache repr"asentiert, 
			folgt Ihren Bewegungen.
\end{itemize}

\bigskip
\begin{center}
	\includegraphics[scale=0.7]{\pfad navigator1} \hfill
	\includegraphics[scale=0.7]{\pfad navigator2} \\
	Ein Klick auf das kleine K"astchen... \hfill ..."offnet den Navigator!
\end{center}

\medskip
\subsection[Beamermodus]{}
\vspace{-4.3ex} {\bf {\large  \qquad \quad \ \ Beamermodus}} 

\icon{icon-beamer}{-3.5}{5.5}{1}

Der Beamermodus ist in erster Linie f"ur die Pr"asentation in Vorlesungen oder "ahnlichen Veranstaltungen gedacht. 
Ist dieser Modus aktiv, so werden die Knoten des Baumes und die Eintr"age des Logbuches vergr"o"sert dargestellt. Der Algorithmustext aus
dem Skript von Prof. Vogler bleibt dabei unver"andert, weil davon ausgegangen wird, dass die interessierten Studenten der 
Vorlesung "uber ein (eventuell aktuelleres) Skript verf"ugen. \\
Sie erreichen den Beamermodus "uber den Men"upunkt {\sc "<AVL-B"aume">-"<Beamermodus">}. Ist der Modus aktiv, so erscheint
neben diesem Men"ueintrag ein H"akchen. Um den Modus wieder auszuschalten, entfernen Sie einfach den Hacken per Klick.

\bigskip

\centerpic{beamermenu}{0.5}{Das Men"u {\sc "<AVL-B"aume">} mit dem Eintrag {\sc "<Beamermodus">}}


\subsection[Hilfe]{}
\vspace{-4.3ex} {\bf {\large  \qquad \quad \ \ Hilfe}} 

\icon{icon-help}{-4.2}{5.5}{1}

Die Hilfe stellt ein wichtiges Nachschlagewerk f"ur all diejenigen dar, die nicht auf Anhieb mit allen Funktionen des 
Moduls klar kommen. Hier k"onnen Sie noch einmal eine genaue Beschreibung der Elemente des Moduls heraussuchen. \\
Sie erreichen die Hilfe "uber den Men"upunkt {\sc "<Hilfe">-"<Hilfe zum AVL-Modul">} oder indem Sie auf den Hilfe-Button
in der Werkzeugleiste klicken.

\medskip
\subsection{Hinweis-Tipps}
Zus"atzlich wird zu den meisten Kontrollelementen, also Buttons, Men"ueintr"age, etc. 
ein kurzer Hinweistext neben dem Mauszeiger bzw. in der Statuszeile des Programmes angezeigt. 
Dies sollte als schnelle Hilfestellung den meisten Anforderungen gen"ugen.
